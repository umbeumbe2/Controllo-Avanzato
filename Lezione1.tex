\chapter{Lezione 1}


\section{Definizione sistema dinamico}

Un sistema dinamico 
è composta da uno stato \(x \in \mathbb{R}^{n}\)
e da una legge di evoluzione
\[\dot{x}(t) = f(x(t), u(t), t)\]

\section{Definizione sistema autonomo}

Un sistema dinamico è detto autonomo se la
sua evoluzione non dipende esplicitamente
dall'ingresso \(u(t)\), cioè:
\[\dot{x}(t) = f(x(t),t)\]

Se il sistema è sia autonomo che tempo invariante
allora si ha:
\[\dot{x}(t) = f(x(t))\]
\section{Definizione stabilità}

Un sistema dinamico 
tempo invariante è detto stabile se 

\[\forall \varepsilon > 0, 
\exists \delta > 0 : x_{0} 
\in B_{\delta}(\bar{x})
\Rightarrow |x_{x_{0}}(t) - \bar{x}| \le \varepsilon, \forall t \geq 0\]

Dove con \(x_{x_{0}}(t)\) si intende 
la traiettoria del sistema 
che parte da \(x_{0}\).


Ora dimostriamo che \(\delta \le 
\varepsilon ,
\forall \varepsilon > 0\).
Supponiamo per assurdo che 
esista un \(\varepsilon >0 \)
per cui si ha  
che la \(\delta\)
per cui è rispettata la 
definizione di stabilità sia
tale che
\(\delta > \varepsilon\).
Allora si ha che per
le 
\(x_{0} \in B_{\delta}(\bar{x}) \backslash
B_{\varepsilon}(\bar{x})\)
vale:
\[|x_{x_{0}}(0) - \bar{x}| 
= \left|x_{0} -\bar{x}\right|> \varepsilon\]
Ma questo è assurdo perché
\(x_{0} 
\in B_{\delta}(\bar{x})\)
dunque rispetta la definizione 
di stabilità per cui
si dovrebbe avere :
\[|x_{x_{0}}(0) - \bar{x}| 
= \left|x_{0} - \bar{x}\right|
\le \varepsilon
\]
Dunque siamo arrivati ad un assurdo.



% ...existing code...
\begin{figure}[h]
  \centering
  \begin{tikzpicture}[scale=1, >=stealth]
    % assi con sole etichette x1 e x2
    \draw[->] (-3,0) -- (3,0) node[right] {$x_1$};
    \draw[->] (0,-3) -- (0,3) node[above] {$x_2$};
    % punto nero con circonferenza in alto a destra
    \filldraw (2,2) circle (1.8pt);
    \draw (2,2) circle (0.35);
  \end{tikzpicture}
  \caption{Assi con etichette $x_1$ e $x_2$.}
  \label{fig:assi-x1-x2}
\end{figure}
% ...existing code...
