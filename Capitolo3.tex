\chapter{Sistemi MIMO nel dominio di Laplace}

Consideriamo il seguente sistema LTI 
a più ingressi e più uscite (MIMO) descritto dalle seguenti equazioni:
\begin{equation}
    \begin{cases}
        \dot{x}(t) = Ax(t) + Bu(t) \\
        y(t) = Cx(t) + Du(t)
    \end{cases}
\end{equation}
Il sistema è MIMO dunque \( u \in \mathbb{R}^{m \times 1}\),
\(y \in \mathbb{R}^{p \times 1}\),
\(B \in \mathbb{R}^{n \times m}\),
\(C \in \mathbb{R}^{p \times n}\)
e \(D \in \mathbb{R}^{p \times m}\).
Il sistema può essere rappresentato nel dominio di Laplace come:
\begin{equation}
    Y(s) = G(s)U(s) 
\end{equation}
Con \( G(s) \in \mathbb{R}^{p \times m} \) matrice di trasferimento del sistema
definita come:
\begin{equation}
    G(s) = C(sI - A)^{-1}B + D
\end{equation}
La matrice \(G(s)\) è composta da \(p \times m\) funzioni di trasferimento
singole \(G_{ij}(s)\):
\[G(s) = \begin{pmatrix}
    G_{11}(s) & \ldots & G_{1m}(s) \\
    \vdots & \vdots & \vdots \\
    G_{p1}(s) & \ldots & G_{pm}(s)
\end{pmatrix}\]
dove ogni elemento \(G_{ij}(s)\) è una funzione polinomiale 
fratta che rappresenta la funzione di trasferimento
tra l'ingresso \(u_j\) e l'uscita \(y_i\).


\section{SVD (Singular Value Decomposition)}
Consideriamo una matrice \(A \in \mathbb{R}^{m \times n}\).
Inoltre dati due vettori \(y \in \mathbb{R}^{m}\) 
e \(x \in \mathbb{R}^{n}\) tali che vale la relazione:
\[y = A x\]
\[A = U \Sigma V^{T}\]
\begin{itemize}
  \item La matrice \(U \in \mathbb{R}^{m \times m}\) (matrice di rotazione delle uscite) 
  è una matrice ortonormale
  composta dai vettori colonna \(u_{1}, u_{2}, \ldots, u_{m}\),
  si ha:
  \[U  =\begin{pmatrix}
    u_{1} & u_{2} & \ldots & u_{m}
  \end{pmatrix}\]
  \item La matrice \(V \in \mathbb{R}^{n \times n}\) 
  (matrice di rotazione degli ingressi) 
  è una matrice ortonormale
  composta dai vettori colonna \(v_{1}, v_{2}, \ldots, v_{n}\),
  si ha:
  \[V  =\begin{pmatrix}
    v_{1} & v_{2} & \ldots & v_{n}
  \end{pmatrix}\]
  \item La matrice \(\Sigma 
  \in \mathbb{R}^{m \times n}\) (matrice dei valori singolari) 

  \[p= min\left\{m,n\right\}\]
   Dove si ha che:
  \[\Sigma =  \begin{pmatrix}
    \Sigma_{1} & 0
  \end{pmatrix} , \hspace{10pt}
    \textnormal{ se } p=m
  \]
  \[\Sigma = \begin{pmatrix}
    \Sigma_{1} \\ 0
  \end{pmatrix} , \hspace{10pt}
    \textnormal{ se } p=n
  \]
\end{itemize}


\[y = \begin{pmatrix}
  u_{1} & u_{2} & \ldots & u_{m} 
\end{pmatrix}
  \begin{pmatrix}
    \Sigma_{1} & 0 & \ldots & 0 \\
    0 & \Sigma_{2} & \ldots & 0 \\
    \vdots & \vdots & \ddots & \vdots \\
    0 & 0 & \ldots & \Sigma_{m} \\
  \end{pmatrix}
\begin{pmatrix}
  v_{1}^{T} \\ 
  v_{2}^{T} \\
  \vdots \\
  v_{n}^{T}
\end{pmatrix}
 \cdot x
\]

Se io prendo \(x=v_{i}\) si ha:
\[
  \begin{pmatrix}
  v_{1}^{T} \\ 
  v_{2}^{T} \\
  \vdots \\
  v_{n}^{T}
\end{pmatrix}
\cdot 
x =
  \begin{pmatrix}
  v_{1}^{T} \\ 
  v_{2}^{T} \\
  \vdots \\
  v_{n}^{T}
\end{pmatrix}
\cdot 
v_{i} 
\]
Ricordando che \(V\) è ortonormale, si che le sue colonne 
sono tra loro ortogonali e di norma 1, dunque si ha:
\[ v^{T}_{j} \cdot v_{i} = 0 , \hspace{10pt} \textnormal{se } i \neq j \]
\[ v^{T}_{j} \cdot v_{i} = 1 , \hspace{10pt} \textnormal{se } i = j \]
Dunque si ha:
\[
  \begin{pmatrix}
  v_{1}^{T} \\ 
  v_{2}^{T} \\
  \vdots \\
  v_{i}^{T} \\
  \vdots \\
  v_{n}^{T}
\end{pmatrix}
\cdot 
v_{i} 
=
  \begin{pmatrix}
  0 \\ 
  0 \\
  \vdots \\
  1 \\
  \vdots \\
  0
\end{pmatrix}
\]
Dove l'uno è nella \(i\)-esima riga.
Andiamo a calcolarci la norma 2 di \(A\):
\[\left|\left|A \right|\right|_{2} = \sqrt{\lambda_{MAX}
\left(A^{T} A\right) } \]
Notiamo che per le proprietà della trasposta \(A^{T}= 
\left( U \Sigma V\right)^{T}  =
 V \Sigma^{T} U^{T} \)
\[ \left|\left|A \right|\right|_{2} =
\sqrt{\lambda_{MAX} \left(V 
\Sigma^{T} U^{T} U \Sigma V^{T}\right)} \]
Ora ricordando che \(U\) è ortonormale si ha \(U^{T} U = I\) 
(\(U^{T} = U^{-1}\)):
\[\left|\left|A \right|\right|_{2} 
=
\sqrt{\lambda_{MAX} \left(V \Sigma^{T} \Sigma  V^{T}\right)}\]
Ora ricordando che \(\Sigma\) 
e \(\Sigma^{T}\) sono diagonali si ha 
che vale la proprietà commutativa tra le matrici, dunque possiamo scrivere:
\[\left|\left|A \right|\right|_{2} 
=
\sqrt{\lambda_{MAX} \left(\Sigma^{T} \Sigma V V^{T}\right)}\]
Ora ricordando che \(\Sigma\) è diagonale vale
\(\Sigma^{T} = \Sigma\), dunque si ha:
\[\left|\left|A \right|\right|_{2} = 
\sqrt{\lambda_{MAX} \Sigma^{T} \Sigma }
= \sqrt{\lambda_{MAX} \Sigma \Sigma }
=  \sqrt{\lambda_{MAX} \Sigma^{2}}
\]
Come si vede la norma 2 di \(A\)
è legata al valore singolare maggiore di \(A\).

\begin{figure}[htbp] % [h] suggerisce a LaTeX di mettere la figura "qui" (here)
    \centering
\begin{tikzpicture}[x=2cm, y=2cm]
        
        % --- ASSI (Mantengo le frecce standard qui) ---
        \draw[thick, ->] (-1.8,0) -- (1.8,0) node[right] {$x_1$};
        \draw[thick, ->] (0,-1.8) -- (0,1.8) node[above] {$x_2$};

        % Cerchio unitario (raggio matematico = 1)
        \draw[thick, dashed, black] (0,0) circle (1);
        \node[black, below right] at (-0.7, -0.7) {$r=1$};

        % --- VETTORI CON NUOVE PUNTE ---
        % Nota: Ho cambiato '->' in '-stealth' per le punte dei vettori

        % Vettore x blu, NORMA MATEMATICA 1, Primo quadrante (60 gradi)
        \draw[-stealth, very thick, blue] (0,0) -- (60:1) node[above right] {$\mathbf{x}$};

        % --- MODIFICA QUI ---
        % Vettore y rosso, NORMA MATEMATICA 1.25 (>1)
        % Spostato nel SECONDO QUADRANTE (es. 135 gradi)
        % x1 negativo, x2 positivo
        \draw[-stealth, very thick, red] (0,0) -- (135:2) node[above left] {$\mathbf{y}$};

    \end{tikzpicture}
    \caption{Esempio di rotazione 
    del vettore di ingresso \(x = v_{1}\)
     nel vettore di uscita 
    amplificato \(y = \sigma_{1} u_{1}\). In questo caso abbiamo preso \(m=n=2\),
    dunque la matrice \(A\) è quadrata, mentre \(\Sigma\) è diagonale. Notiamo
    che abbiamo potuto rappresentare tutto sullo stesso grafico solo 
    perché \(m=n\).}
    \label{fig:cerchio_assi}
\end{figure}


\section{Guadagno}
Definiamo gli spazi \(L^{p}\)
come gli spazi delle funzioni \(p\)-sommabili,
cioè in cui esiste la norma \(p\)-esima finita:
\[L^{p}(\mathbb{R^{n}}) = 
\left\{f : \mathbb{R} \to \mathbb{R}^{n} : 
\int_{-\infty}^{+\infty} \left|\left|f(\tau)
\right|\right|^{p}_{p} \, d\tau < \infty \right\}\]
Dove essendo \(f\) una funzione vettoriale a variabile reale
la norma \(p\)-esima è definita come:
\[\left|\left|f(\tau)\right|\right|_{p} =
\left( \sum_{i=1}^{n} |f_{i}(\tau)|^{p} \right)^{\frac{1}{p}}\]
Dove con \(f_{i}(\tau)\) si intende la \(i\)-esima componente del vettore
\(f(\tau)\) (che è un vettore di \(n\) componenti in cui ognuna è una funzione 
di \(\tau\))
\section{Poli e zeri per sistemi MIMO }
Ora ci poniamo il problema di definire poli e zeri per sistemi MIMO.
Per i sistemi SISO i poli sono definiti come gli zeri del denominatore della 
funzione di trasferimento, mentre gli zeri sono definiti come gli zeri del numeratore
della funzione di trasferimento. Il problema è che per i sistemi MIMO
ho una matrice di trasferimento \(G(s)\) composta da \(pm\) funzioni 
di trasferimento singole \(G_{ij}(s)\), quindi bisogna dare una nuova 
definizione di poli e zeri per sistemi MIMO.
Per prima cosa diamo la definizione di polo per sistemi MIMO.

\begin{definizione}
    Un polo di un sistema MIMO è un valore di \(s\) tale che la matrice \(G(s)\)
    diventa singolare, cioè il suo determinante è nullo, dunque per quel valore di \(s\)
    la matrice \(G(s)\) non è invertibile.
\end{definizione}


\begin{definizione}
    Si definisce rango nominale di una matrice di polinomi 
    \(G(s)\), il rango della matrice che si ha per tutti i valori di 
    \(s\) eccetto un numero finito di valori.
\end{definizione}


\begin{definizione}
    Uno zero di un sistema MIMO è un valore di \(s\) tale 
    che la matrice \(G(s)\)
    perde rango, cioè il rango della matrice \(G(s)\) 
    diminuisce rispetto al suo rango nominale (quello che ha per 
    tutti gli altri valori di \(s\)).
\end{definizione}

