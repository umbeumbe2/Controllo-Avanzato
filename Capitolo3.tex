\chapter{Stabilità Robusta}

Supponiamo ora che il nostro sistema dinamico sia affetto da
inceertezze parametriche, cioè che il modello matematico (dunque la matrice 
delle dinamica \(A\)) abbia delle variazioni intorno al suo valore nominale.
In questo caso il sistema dinamico si può scrivere come
\begin{equation}
    \dot{x}(t) = (A(p(t)))x(t)
\end{equation}
dove \(p(t) \) è un parametro incerto che varia nel tempo
ed appartiene ad un insieme \(\mathcal{P}\).
In generale si studiano sistemi di 3 tipologie:
\begin{itemize}
    \item \textbf{Sistemi Stazionari} : in cui il parametro \(p(t) = p \in \mathcal{P}\) non 
    è funzione del tempo (ma rimane incerto).
    \item \textbf{Sistemi Tempo-Varianti} : in cui il parametro \(p(t) \in \mathcal{P}\) 
    varia nel tempo in modo arbitrario.
    \item \textbf{Sistemi Quasi Stazionari} : in cui il parametro \(p(t) \in \mathcal{P}\)
    varia nel tempo in modo "lento" ( caso per caso si definisce cosa significa lento).
\end{itemize}
Noi studieremo solo i sistemi stazionari e tempo-varianti.