\documentclass[a4paper, openany]{book}
\usepackage{blindtext}
\usepackage[a4paper, total={6in, 8in}, margin=0.8in]{geometry}
\usepackage[center]{titlesec}
\usepackage[italian]{babel}
\usepackage{amsmath}
\usepackage{amsthm}
\usepackage{amsfonts}
\usepackage{amssymb}
\usepackage{manyfoot}
\usepackage{mathtools, nccmath}
\usepackage{interval}
\usepackage{listings}
\usepackage{subcaption}


\usepackage[dvipsnames]{xcolor}

\usepackage{tikz}
\usepackage{pgfplots}

\usetikzlibrary{shapes,arrows,positioning} 
\usetikzlibrary{shadows,decorations.pathmorphing}
\usetikzlibrary{calc}
\usetikzlibrary{patterns}
\usetikzlibrary{decorations.pathmorphing, arrows.meta, positioning, shapes.misc}
\usetikzlibrary{decorations.markings}


% Definizione del linguaggio MATLAB
\lstdefinelanguage{Matlab}{
  keywords={break,case,catch,continue,else,elseif,end,for,function,
            global,if,otherwise,persistent,return,switch,try,while},
  sensitive=true,
  comment=[l]\%,             % commenti con %
  morestring=[m]',           % stringhe in apici singoli
}

% Stile per i blocchi
\lstset{
  language=Matlab,
  basicstyle=\ttfamily\small,
  keywordstyle=\color{blue}\bfseries,
  commentstyle=\color{gray}\itshape,
  stringstyle=\color{teal},
  frame=single,
  backgroundcolor=\color{black!5},
  showstringspaces=false,
  breaklines=true,
  captionpos=b
}
\bibliographystyle{unsrt}



% 1. Definizione dello stile per Teoremi e Lemmi (Titolo grassetto, corpo corsivo)
\theoremstyle{plain}
\newtheorem{teorema}{Teorema}[section] % La numerazione riparte ad ogni sezione
\newtheorem{lemma}[teorema]{Lemma}     % Il lemma usa lo stesso contatore del teorema

% 2. Definizione dello stile per le Definizioni (Titolo grassetto, corpo normale)
\theoremstyle{definition}
\newtheorem{definizione}[teorema]{Definizione} % Condivide la numerazione con teorema




\pgfplotsset{width=10cm,compat=1.9}

% We will externalize the figures

\DeclareNewFootnote{R}[roman]
\renewcommand{\footnoterule}
    {\noindent\smash{\rule[3pt]{\textwidth}{0.4pt}}}


\theoremstyle{remark}
\swapnumbers

\pagestyle{plain}
\pagenumbering{arabic}


%Per non usare /noindent
\setlength\parindent{0pt}

\usepackage{enumitem}
\setlist[itemize]{noitemsep, topsep=0pt, parsep=2pt, partopsep=0pt, itemsep=0pt}

\usepackage{pdfpages} % nel preambolo

\usepackage{caption}
\captionsetup{font=small, skip=4pt}      % meno spazio tra figura e didascalia
\usepackage{placeins}                   % \FloatBarrier per forzare il flush dei float
% Riduci gli spazi verticali dei float
\setlength{\textfloatsep}{8pt plus 2pt minus 2pt}
\setlength{\floatsep}{6pt plus 1pt minus 1pt}
\setlength{\intextsep}{6pt plus 1pt minus 1pt}

\begin{document}



\tableofcontents
\linespread{1.3}

\large

\chapter{Stabilità nei sistemi dinamici}


\section{Sistema dinamico e punti di equilibrio}

Un sistema dinamico 
è composta da uno stato \(x \in \mathbb{R}^{n}\)
e da una legge di evoluzione

\begin{equation}
  \dot{x}(t) = f(x(t), u(t), t)
  \label{sistema-dinamico-generale}
\end{equation}

\begin{definizione}
Un sistema è detto \textbf{tempo invariante} se la
  sua evoluzione non dipende esplicitamente dal tempo \(t\), cioè:
  \[\dot{x}(t) = f(x(t), u(t))\]
\end{definizione}


\begin{definizione}
Un sistema è detto \textbf{autonomo} se la sua evoluzione non dipende esplicitamente
  dall'ingresso \(u(t)\), cioè:
  \[\dot{x}(t) = f(x(t),t)\]
\end{definizione}

Se il sistema è sia autonomo che tempo invariante
allora si ha:
\[\dot{x}(t) = f(x(t))\]




\begin{definizione}
Dato un sistema dinamico \textbf{tempo invariante} nella forma:
\[\dot{x}(t) = f(x(t), u(t))\]
un punto di equilibrio \((\bar{x}, \bar{u})\) è una coppia 
stato-ingresso tale che:
\[\dot{x} = f(\bar{x}, \bar{u}) = 0\]


\end{definizione}



\section{Stabilità e Convergenza}



\begin{definizione}

Un punto di equilibrio \(\bar{x}\) di un sistema dinamico 
tempo invariante è detto stabile se:

\begin{equation}
  \forall \varepsilon > 0, 
  \exists \delta > 0 : x_{0} 
  \in B_{\delta}(\bar{x})
  \Rightarrow |x_{x_{0}}(t) - \bar{x}| \le \varepsilon, \forall t \geq 0
  \label{stabilità}
\end{equation}Dove con \(x_{x_{0}}(t)\) si intende 
la traiettoria del sistema 
che parte da \(x_{0}\).
\end{definizione}



Ora dimostriamo che \(\delta \le 
\varepsilon ,
\forall \varepsilon > 0\).
\begin{proof}
  
Supponiamo per assurdo che 
esista un \(\varepsilon >0 \)
per cui si ha  
che la \(\delta\)
per cui è rispettata la 
definizione di stabilità sia
tale che
\(\delta > \varepsilon\).
Allora si ha che per
le 
\(x_{0} \in B_{\delta}(\bar{x}) \backslash
B_{\varepsilon}(\bar{x})\)
vale:
\[|x_{x_{0}}(0) - \bar{x}| 
= \left|x_{0} -\bar{x}\right|> \varepsilon\]
Ma questo è assurdo perché
\(x_{0} 
\in B_{\delta}(\bar{x})\)
dunque rispetta la definizione 
di stabilità per cui
si dovrebbe avere :
\[|x_{x_{0}}(0) - \bar{x}| 
= \left|x_{0} - \bar{x}\right|
\le \varepsilon
\]
Dunque siamo arrivati ad un assurdo.

\end{proof}



\begin{definizione}
  Dato un sistema dinamico un punto di equilibrio
  \(\bar{x}\) è detto \textbf{convergente} se esiste un 
  \(\delta >0\) tale che per ogni condizione iniziale
  \(x_{0} \in B_{\delta}(\bar{x})\) si ha che:
  \[\lim_{t \to \infty} \left|\left|x(t) - \bar{x}\right|\right| = 0\]

\end{definizione}

\begin{definizione}
  Un punto di equilibrio \(\bar{x}\) è detto 
  \textbf{isolato} se esiste un intorno di \(\bar{x}\)
  che non contiene altri punti di equilibrio.
\end{definizione}


% ...existing code...
\begin{figure}[htbp]
  \centering
  \begin{tikzpicture}[scale=1, >=stealth]
    % assi con sole etichette x1 e x2
    \draw[->] (-3,0) -- (3,0) node[right] {$x_1$};
    \draw[->] (0,-3) -- (0,3) node[above] {$x_2$};


    % cerchio grande e raggio epsilon
    \draw (2,2) circle (1.5);
    \draw[thick, blue] (2,2) -- ++(1.5,0) node[midway, above] {$\varepsilon$};

    % cerchio piu' piccolo rosso trasparente e raggio delta (più corto)
    \filldraw[fill=red, fill opacity=0.25, draw=red] (2,2) circle (0.6);
    \draw[thick, red] (2,2) -- ++(-0.6,-0) node[midway, above] {$\delta$};

    \filldraw[ForestGreen] (2,2) circle (1.8pt) node[below=3pt] {$\bar{x}$};

  \end{tikzpicture}
  \caption{In \textcolor{ForestGreen}{verde} si ha il 
  punto di equilibrio \textcolor{ForestGreen}{\(\bar{x}\)}. 
  In rosso si ha l'insieme delle condizioni iniziali per cui le traiettorie rimangono
  confinate  all'interno del cerchio di raggio \(\varepsilon\).}
  \label{fig:assi-x1-x2}
\end{figure}
% ...existing code...
\section{Stabilità nei sistemi lineari}

Consideriamo un sistema lineare
tempo invariante:
\[\dot{x}(t) = Ax(t) + Bu(t)\]
Dove \(A \in \mathbb{R}^{n \times n}\) e 
\(B \in \mathbb{R}^{n \times m}\).
Per trovare i punti di equilibrio dobbiamo imporre:
\[\dot{x}(t) = 0 \Longleftrightarrow Ax(t) + Bu(t) = 0\]
Dunque i punti di equilibrio 
sono le coppie \((\bar{x}, \bar{u})\) che soddisfano:
\[ 
A\bar{x} + B\bar{u} = 0
\]
Notiamo subito che il punto \((\bar{x}, \bar{u}) = (0,0)\)
 è sicuro un punto di equilibrio.
Mentre gli altri punti si trovano risolvendo il sistema lineare
omogeneo considerando \(\bar{x}\) e \(\bar{u}\) come incognite.
Se invece di considerare le coppie \((\bar{x}, \bar{u})\) di equilibrio
consideriamo solo gli stati di equilibrio \(\bar{x}\) con un 
fissato \(\bar{u}\), allora l'unica incognita è \(\bar{x}\) 
mentre \(B \bar{u}\) è un termine noto.
In questo caso il sistema lineare da risolvere è:
\[A \bar{x} = - B \bar{u}\]
Questo sistema al variare di \(\bar{u}\) 
(che funge da parametro) ammette soluzioni differenti,
però la matrice \(A\) ci dice quante sono queste soluzioni.
Se \(A\) è invertibile (ovvero \(\det(A) \neq 0\))
allora esiste un'unica soluzione per ogni \(\bar{u}\):
\[\bar{x} = -A^{-1}B \bar{u}\]
Se invece \(A\) non è invertibile (ovvero \(\det(A) = 0\))
allora il sistema può ammettere infinite soluzioni oppure nessuna soluzione
(ricordiamo che il sistema si risolve
al variare del parametro \(\bar{u}\)).

\section{Sistema nel dominio di Laplace}
Consideriamo il sistema lineare tempo invariante:
\[\dot{x}(t) = Ax(t) + Bu(t)\]
Applichiamo la trasformata di Laplace ambo i membri otteniamo:
\[sX(s) - x(0) = AX(s) + BU(s)\]
Portando \(x(0)\) a destra e portando \(AX(s)\) a sinistra otteniamo:
\[sX(s) - AX(s) = BU(s) + x(0)\]
Ricordando che \(sX(s) = s I X(s)\) e mettendo in evidenza \(X(s)\) otteniamo:
\[(sI - A)X(s) = BU(s) + x(0)\]
Ora premoltiplichiamo ambo i membri per
\((sI - A)^{-1}\) otteniamo:
\[X(s) = \textcolor{red}{(sI - A)^{-1}BU(s)} + \textcolor{blue}{(sI - A)^{-1}x(0)}\]
Dove in \textcolor{red}{rosso} abbiamo la parte dovuta all'ingresso
che prende il nome di \textcolor{red}{evoluzione forzata}  dello stato, 
mentre in \textcolor{blue}{blu}
abbiamo la parte dovuta alla  \textcolor{blue}{evoluzione libera}
dello stato (che ricordiamo essere nulla se \(x(0) = 0\)).
Andando a fare la stessa cosa per l'equazione di uscita:
\[y(t) = Cx(t) + Du(t)\]
Applichiamo la trasformata di Laplace otteniamo:
\[Y(s) = CX(s) + DU(s)\]
Sostituendo \(X(s)\) otteniamo:
\[Y(s) = C\left[(sI - A)^{-1}BU(s) + (sI - A)^{-1}x(0)\right] + DU(s)\]
Dove riusciamo un'altra volta a distinguere una parte
in \textcolor{red}{evoluzione forzata} ed un una in \textcolor{blue}{
evoluzione libera}:
\[Y(s) =\textcolor{red}{C \left[\left(sI - A\right)^{-1} + D\right]BU(s)} +
\textcolor{blue}{ 
\left(sI - A\right)^{-1}x(0)} \]


\section{Stabilità tramite l'approccio alla Lyapunov}

Ora ci interessa introdurre un metodo per studiare la stabilità
dei sistemi dinamici che prende il nome di \textbf{approccio di Lyapunov},
che ci permette di studiare la stabilità dei sistemi andandoci a 
trovare delle specifiche funzioni 
scalari \(V(x) : 
\mathbb{R}^n \to \mathbb{R}\) dette \textbf{funzioni di Lyapunov}.
Per prima cosa ricordiamo che una funzione \(V(x) : \mathbb{R}^{n} 
\to \mathbb{R}\) e un suo punto di equilibrio 
possono essere classificati come:
\begin{itemize}
  \item \textbf{definita positiva} 
  in \(\bar{x}\) se \(V(\bar{x}) = 0\) ed esiste un \(\delta > 0\)
  tale che \(V(x) > 0\) (notiamo il maggiore stretto) per ogni \(x \in B_{\delta}(\bar{x})\)
  \item \textbf{semidefinita positiva} 
  in \(\bar{x}\) se \(V(\bar{x}) = 0\) ed esiste un \(\delta > 0\)
  tale che \(V(x) \ge 0\) per ogni \(x \in B_{\delta}(\bar{x})\)
  \item \textbf{definita negativa} 
  in \(\bar{x}\) se \(V(\bar{x}) = 0\) ed esiste un \(\delta > 0\)
  tale che \(V(x) < 0\) per ogni \(x \in B_{\delta}(\bar{x})\)
  \item \textbf{semidefinita negativa} 
  in \(\bar{x}\) se \(V(\bar{x}) = 0\) ed esiste un \(\delta > 0\)
  tale che \(V(x) \le 0\) per ogni \(x \in B_{\delta}(\bar{x})\)
\end{itemize}
Queste proprietà diventano \textbf{globali}
quando sono vere per ogni \(\bar{x} \in \mathbb{R}^{n}\).
La potenza del metodo che stiamo per introdurre
sta nel fatto che è valida anche per sistemi non lineari,
nel momento in cui aggiungeremo l'ipotesi di linearità andremo 
ad ottenere anche altri risultati.

\begin{teorema}
Teorema di Lyapunov.  Consideriamo il sistema dinamico:
\[\dot{x}(t) = f(x(t))\]
Consideriamo il punto di equilibrio \(\bar{x}\) per \(f\).
Se \(f\) è definita, è continua ed anche la sua derivata è continua
in un intorno \(D\) di \(\bar{x}\),
cioè \(f \in C^{1}(D)\) (\(f\) 
è un vettore di funzioni quindi quando diciamo che è 
continua intendiamo che ogni sua componente lo è).
Se esiste una funzione \(V(x) : \mathbb{R}^{n} \to \mathbb{R}\), 
continua e derivabile in \(D\), tale che:
\begin{itemize}
  \item \(V(x)\) è definita positiva in \(\bar{x}\), cioè:
  \[\begin{cases}
    V(\bar{x}) = 0 \\
    \exists \delta : V(x) > 0, \quad \forall x \in B_{\delta}(\bar{x}) \backslash \{\bar{x}\}
  \end{cases}\]
  \item \(\dot{V}(x)\) è semidefinita negativa in \(\bar{x}\), cioè:
  \[\begin{cases}
    V(\bar{x}) = 0 \\
    \exists \delta > 0: \dot{V}(x) = \nabla V(x) \cdot 
    \displaystyle 
    \frac{dx}{dt} = \nabla V(x(t)) f(x(t)) \le 0, \quad \forall x \in B_{\delta}(\bar{x}),
    \forall t > t_0 
  \end{cases}\]
\end{itemize}
Allora si ha che il punto di equilibrio \(\bar{x}\) è stabile.
\end{teorema}

\begin{proof}
Noi vogliamo dimostrare che \(\bar{x}\) è un punto di equilibrio stabile,
cioè che per ogni \(\varepsilon > 0\) esiste un \(\delta > 0\)
tale che se \(x_{0} \in B_{\delta}(\bar{x})\) allora
la traiettoria \(x(t)\) che parte da \(x_{0}\)
rimane confinata in \(B_{\varepsilon}(\bar{x})\).
Dunque partiamo con il fissare un generico \(\varepsilon > 0\).
Prendiamo la curva di livello (della funzione \(V(x)\)) 
a valore maggiore e completamente contenuta in \(\bar{B}_{\varepsilon}(\bar{x})\) 
(con il trattino sopra intendiamo la chiusura), come 
in Figura \ref{fig:lyapunov-proof}. Ora chiamiamo \(\delta\)
la distanza minima tra \(\bar{x}\) e la curva di livello di valore \(\bar{V}\).
Ora consideriamo una \(x(t)\)
che parte da un punto \(x_{0} \in B_{\delta}(\bar{x})\) 
\((x(t_{0}) = x_{0})\). A noi interessa dimostrare che \(\bar{x}\) è
stabile, cioè che \(\forall x_0 \in B_{\delta}(\bar{x})\)
la traiettoria \(x(t)\) rimane confinata all'interno di \(B_{\varepsilon}(\bar{x})\)
\(\forall t \ge 0\). Per dimostrare che \(x(t)\) rimane confinata in
\(B_{\varepsilon}(\bar{x})\) distinguiamo due casi:
\begin{itemize}
  \item Nel caso in cui si ha \(\dot{V}(x) = 0\) la traiettoria \(x(t)\) nel peggiore 
  casi rimane confinata sulla curva di livello a valore \(\bar{V}\), o su un'altra 
  curva di livello a valore minore di \(\bar{V}\) (dunque rimane confinata in
  \(B_{\varepsilon}(\bar{x})\)).
  \item  Se  \(\dot{V}(x) < 0\) allora la traiettoria \(x(t)\) si muove 
  in modo tale da far diminuire il valore di \(V(x(t))\), quindi si muove
   verso l'interno   (essendo \(\bar{V}\) la curva di livello a valore massimo)
  della curva di livello a valore \(\bar{V}\) 
  (dunque rimane confinata in \(B_{\varepsilon}(\bar{x})\))
\end{itemize}
Quindi in entrambi i casi la traiettoria \(x(t)\)
rimane confinata in \(B_{\varepsilon}(\bar{x})\). 
Dunque abbiamo dimostrato che per ogni \(\varepsilon > 0\)
esiste un \(\delta > 0\) tale che se \(x_{0} \in B_{\delta}(\bar{x})\)
allora la traiettoria \(x(t)\) rimane confinata in
\(B_{\varepsilon}(\bar{x})\) \(\forall t \ge t_{0}\),
ma questa è proprio la definizione di stabilità. 
\begin{figure}[htbp]
    \centering
    \begin{tikzpicture}[>=stealth, scale=1.3]

        % 1. Parametri geometrici
        \coordinate (XBAR) at (3, 2.5);
        
        % Raggio Epsilon (Cerchio esterno limite)
        \def\Reps{2.5}
        % Raggio Delta (Cerchio interno)
        \def\Rdelta{1.0} 

        % 2. Assi
        \draw[->, thick] (-1, 0) -- (7, 0) node[right] {$x_1$};
        \draw[->, thick] (0, -1) -- (0, 6) node[above] {$x_2$};

        % 3. Centro x_bar
        \fill (XBAR) circle (2pt) node[below left] {$\bar{x}$};

        % 4. Circonferenza Epsilon (Nera - Esterna)
        \draw[thick] (XBAR) circle (\Reps);
        \draw[dashed, ->] (XBAR) -- ++(45:\Reps) node[midway, above left] {$\epsilon$};
        
        % 5. Curva di livello V_bar di forma CASUALE (Blu)
        % Utilizziamo 'plot smooth cycle' definendo dei punti di controllo attorno a XBAR.
        % L'importante è che la distanza di questi punti da XBAR sia sempre < Reps (2.5)
        % Usiamo coordinate polari relative: ([shift=(angolo:raggio)]Centro)
        \draw[blue, thick] plot [smooth cycle, tension=0.7] coordinates {
            ([shift=(0:2.2)]XBAR)   % Est, vicino al bordo
            ([shift=(50:1.5)]XBAR)  % Nord-Est, rientranza
            ([shift=(90:2.1)]XBAR)  % Nord
            ([shift=(140:1.7)]XBAR) % Nord-Ovest
            ([shift=(180:2.3)]XBAR) % Ovest, vicino al bordo
            ([shift=(230:1.6)]XBAR) % Sud-Ovest
            ([shift=(270:1.9)]XBAR) % Sud
            ([shift=(320:1.4)]XBAR) % Sud-Est, rientranza marcata
        };
        % Etichetta V_bar
        \node[blue, right] at ($(XBAR) + (2.2, 0.2)$) {$\bar{V}$};

        % 6. Circonferenza Delta (Rossa - Interna)
        % Contenuta nella curva casuale
        \draw[red, thick] (XBAR) circle (\Rdelta);
        \draw[red, dashed, ->] (XBAR) -- ++(135:\Rdelta) node[midway, right] {$\delta$};

    \end{tikzpicture}
    \caption{Visualizzazione con una curva di livello $\bar{V}$ (blu) di forma generica non ellittica, strettamente contenuta all'interno della circonferenza di raggio $\epsilon$ (nera).}
    \label{fig:livello_casuale}
\end{figure}

\end{proof}


Se \( f \in C^{1}(D) \) allora è anche sicuramente
 localmente Lipschitziana
in  \(D\) (poichè se una funzione è \(C^{1}\) allora 
il suo gradiente è limitato) cioè:
\[\exists L >0  : \|f(x) - f(y)\| \le L \|x - y\|, \quad \forall x,y \in I\] 

\section{Lyapunov per Sistemi LTI}
Per i sistemi lineari tempo invarianti possiamo 
utilizzare il metodo di Lyapunov per studiare l'asintotica stabilità
del sistema. 
Consideriamo il sistema LTI:
\[\dot{x} = Ax + Bu\]
Ricordiamo che per tutti i sistemi linerari l'asintotica stabilità è una proprietà 
del sistema, cioè se un punto di equilibrio è asintoticamente stabile allora lo sono tutti.
Tutti i sistemi LTI ammettono come punto di equilibrio l'origine
\((\bar{x}, \bar{u}) = (0,0)\), 
e quindi possiamo anche andare a studiare direttamente 
la stabilità di questo punto di equilibrio (che ci 
permette di non considerare gli ingressi). Dunque consideriamo direttamente
il sistema autonomo (che si evolve senza ingressi):
\[\dot{x} = Ax\]
Studiamo con il metodo di Lyapunov la stabilità dell'origine (nello spazio 
dello stato e degli ingressi) come 
punto di equilibrio. Per i sistemi LTI vale il seguente teorema:
\begin{teorema}
Teorema di Lyapunov per sistemi LTI. 
Il sistema LTI:
\[\dot{x} = Ax\]
è stabile se e solo se per ogni matrice \(Q \) simmetrica definita positiva (\(Q=Q^{T} > 0\)),
esiste una matrice simmetrica definita positiva \(P=P^{T} > 0\)
tale che:
\begin{equation}  
  A^{T}P + PA = -Q
\end{equation}
quest'ultima equazione prende il nome di \textbf{equazione di Lyapunov}.
\end{teorema}
Dimostriamo il seguente teorema.
\begin{proof}
  Partiamo con il mostrare che l'esistenza di \(P\) 
  che soddisfa l'equazione di Lyapunov è una \textbf{condizione necessaria}
  alla asintotica stabilità. Prendiamo una generica \(Q=Q^{T}>0\), 
  vogliamo dimostrare che supposto il sistema sia asintoticamente stabile, allora deve 
  esistere una \(P=P^{T}>0\) che soddisfa l'equazione di Lyapunov:
  \[  A^{T}P + PA = -Q \]
  Consideriamo la \(P\) definita come:
  \[ P =\int_{0}^{+\infty} e^{A^{T}t} Q e^{At} dt \]
  Per prima cosa diciamo che questa \(P\) è ben definita (l'integrale converge 
  \(\forall Q\)) poichè il sistema è asintoticamente stabile, dunque 
  gli esponenziali che usciranno nei calcoli avranno tutti 
  esponenti negativi, dunque si avrà \(e^{\lambda t} \to 0\) per \(t \to +\infty\)
  (poichè \(\lambda <0 \)).

  Per costruzione questa \(P\) è simmetrica.
  Dimostriamo ora che questa \(P\) è definita positiva. Preso un generico
  \(x \in \mathbb{R}^{n} - \left\{0\right\}\) vogliamo dimostrare che \(x^{T}Px > 0\).
  Calcoliamo:
  \[ x^{T}Px = x^{T} \left( \int_{0}^{+\infty} e^{A^{T}t} Q e^{At} dt \right) x  \]
  Portiamo le \(x\) dentro l'integrale:
  \[ x^{T}Px = \int_{0}^{+\infty} x^{T} e^{A^{T}t} Q e^{At} x dt \]
  Notiamo ora che a primo membro abbiamo \(x^{T} e^{A^{T} t} = \left(e^{At} x\right)^{T}\):
  \[ x^{T}Px = \int_{0}^{+\infty} \left(e^{At} x\right)^{T} Q \left(e^{At} x\right) dt\]
  Ora poniamo \(y = e^{At}x\), ci interessa dimostrare che:
  \[ \int_{0}^{+\infty} y^{T} Q y dt > 0 \]
  Ora noi sappiamo che \(Q\) è definita positiva,
  dunque per definizione di matrice definita positiva si ha che:
  \[ y^{T} Q y > 0 ,  \forall y \neq 0 \]
  Il nostro problema è diventato dunque dimostrare che \(y \neq 0\) per ogni
  \(x \neq 0\). Ricordiamo che l'esponenziale di 
  matrice è sempre invertibile, dunque la dimensione del \(ker(e^{At})\) è nulla,
  quindi \(y =e^{At}x \neq 0 \forall x \neq 0\). Dunque abbiamo dimostrato che:
  \[ y^{T} Q y > 0 ,  \forall x \neq 0 \]
  Quindi abbiamo dimostrato che l'integrando è sempre positivo per ogni \(t \ge 0\).
  Ma se l'integrando è sempre positivo anche l'intergrale è
  positivo dunque la matrice \(P\) è definita positiva.
  Ora dimostriamo che questa \(P > 0\) soddisfa l'equazione di Lyapunov.
  Calcoliamo:
  \[A^{T} P + P A = 
  A^{T} \int_{0}^{+\infty} e^{A^{T}t} Q e^{At} dt  +
  \int_{0}^{+\infty} e^{A^{T}t} Q e^{At} dt  A\] 
  Portando \(A^{T}\) e \(A\) dentro gli integrali, 
  ed unendo gli integrali otteniamo:
    \[A^{T} P + P A = 
  \int_{0}^{+\infty}
  A^{T} 
  e^{A^{T}t} Q e^{At} dt  +
  e^{A^{T}t} Q e^{At} dt  
  A\] 
  Ora notiamo che il termine all'interno dell'integrale
  è la derivata di \(e^{A^{T}t} Q e^{At}\) rispetto a \(t\), dunque possiamo riscrivere:
  \[A^{T} P + P A =  
  \int_{0}^{+\infty} \frac{d}{dt} \left( e^{A^{T}t} Q e^{At} \right) dt
  \]
  Ma l'integrale della derivata è proprio la funzione valutata agli estremi:
  \[ 
  A^{T} P + P A= \left[ e^{A^{T}t} Q e^{At} \right]_{0}^{+\infty}
  \]
  Ma \[e^{A^{T}t} Q e^{At}\] valutata in \(0\) vale \(Q\), mentre
  valutata in \(+\infty\) vale \(0\) (poichè il sistema è asintoticamente stabile),
  dunque si ha:
  \[ A^{T} P + P A = 0 - Q = -Q\]
  Dunque abbiamo dimostrato che se il sistema è asintoticamente stabile
  allora per ogni \(Q=Q^{T} > 0\) esiste una \(P=P^{T} > 0\)
  che soddisfa l'equazione di Lyapunov. \\
  Ora dimostriamo la \textbf{condizione sufficiente}, cioè
  se \(\forall Q=Q^{T} >0\) esiste una \(P=P^{T} > 0\) che soddisfa l'equazione di Lyapunov
  allora il sistema è asintoticamente stabile.
  Per ipotesi dunque sappiamo che fissata una \(Q=Q^{T} > 0\) 
  esiste almeno una \(P=P^{T} > 0\) che soddisfa l'equazione di Lyapunov,cioè:
  \[ A^{T} P + P A = -Q \]
  Il nostro obiettivo è dimostrare che il sistema è asintoticamente stabile,
  dunque andiamo a considerare la funzione di Lyapunov:
  \[ V(x) = x^{T} P x \]
  Noi sappiamo che \(P\) è definita positiva, dunque \(V(x)\) è definita positiva.
  Calcoliamo ora la derivata di \(V(x)\) lungo
  le traiettorie del sistema (ricordiamo la regola 
  della derivata di una forma quadratica):
  \[ \dot{V}(x) = \dot{x}^{T} P x + x^{T} P \dot{x} \]
  Ricordiamo che \(\dot{x} = Ax\), dunque sostituiamo:
  \[ \dot{V}(x) = (Ax)^{T} P x + x^{T} P (Ax) \]
  Facendo il trasposto a primo membro:
  \[ \dot{V}(x) = x^{T} A^{T} P x + x^{T} P A x \]
  Mettiamo in evidenza \(x^{T}\) e \(x\):
  \[ \dot{V}(x) = x^{T} (A^{T} P + P A) x \]
  Ora ricordiamo che \(A^{T} P + P A = -Q\), dunque sostituiamo:
  \[ \dot{V}(x) = x^{T} (-Q) x \]
  Ma \(Q\) è definita positiva, dunque \(-Q\)
  è definita negativa, dunque anche \(\dot{V}(x)\) è definita negativa.
  Quindi il sistema è asintoticamente stabile per il teorema di Lyapunov,
  con funzione di Lyapunov \(V(x) = x^{T} P x\).
\end{proof}


\section{SVD (Singular Value Decomposition)}
Consideriamo una matrice \(A \in \mathbb{R}^{m \times n}\).
Inoltre dati due vettori \(y \in \mathbb{R}^{m}\) 
e \(x \in \mathbb{R}^{n}\) tali che vale la relazione:
\[y = A x\]
\[A = U \Sigma V^{T}\]
\begin{itemize}
  \item La matrice \(U \in \mathbb{R}^{m \times m}\) (matrice di rotazione delle uscite) 
  è una matrice ortonormale
  composta dai vettori colonna \(u_{1}, u_{2}, \ldots, u_{m}\),
  si ha:
  \[U  =\begin{pmatrix}
    u_{1} & u_{2} & \ldots & u_{m}
  \end{pmatrix}\]
  \item La matrice \(V \in \mathbb{R}^{n \times n}\) 
  (matrice di rotazione degli ingressi) 
  è una matrice ortonormale
  composta dai vettori colonna \(v_{1}, v_{2}, \ldots, v_{n}\),
  si ha:
  \[V  =\begin{pmatrix}
    v_{1} & v_{2} & \ldots & v_{n}
  \end{pmatrix}\]
  \item La matrice \(\Sigma 
  \in \mathbb{R}^{m \times n}\) (matrice dei valori singolari) 

  \[p= min\left\{m,n\right\}\]
   Dove si ha che:
  \[\Sigma =  \begin{pmatrix}
    \Sigma_{1} & 0
  \end{pmatrix} , \hspace{10pt}
    \textnormal{ se } p=m
  \]
  \[\Sigma = \begin{pmatrix}
    \Sigma_{1} \\ 0
  \end{pmatrix} , \hspace{10pt}
    \textnormal{ se } p=n
  \]
\end{itemize}


\[y = \begin{pmatrix}
  u_{1} & u_{2} & \ldots & u_{m} 
\end{pmatrix}
  \begin{pmatrix}
    \Sigma_{1} & 0 & \ldots & 0 \\
    0 & \Sigma_{2} & \ldots & 0 \\
    \vdots & \vdots & \ddots & \vdots \\
    0 & 0 & \ldots & \Sigma_{m} \\
  \end{pmatrix}
\begin{pmatrix}
  v_{1}^{T} \\ 
  v_{2}^{T} \\
  \vdots \\
  v_{n}^{T}
\end{pmatrix}
 \cdot x
\]

Se io prendo \(x=v_{i}\) si ha:
\[
  \begin{pmatrix}
  v_{1}^{T} \\ 
  v_{2}^{T} \\
  \vdots \\
  v_{n}^{T}
\end{pmatrix}
\cdot 
x =
  \begin{pmatrix}
  v_{1}^{T} \\ 
  v_{2}^{T} \\
  \vdots \\
  v_{n}^{T}
\end{pmatrix}
\cdot 
v_{i} 
\]
Ricordando che \(V\) è ortonormale, si che le sue colonne 
sono tra loro ortogonali e di norma 1, dunque si ha:
\[ v^{T}_{j} \cdot v_{i} = 0 , \hspace{10pt} \textnormal{se } i \neq j \]
\[ v^{T}_{j} \cdot v_{i} = 1 , \hspace{10pt} \textnormal{se } i = j \]
Dunque si ha:
\[
  \begin{pmatrix}
  v_{1}^{T} \\ 
  v_{2}^{T} \\
  \vdots \\
  v_{i}^{T} \\
  \vdots \\
  v_{n}^{T}
\end{pmatrix}
\cdot 
v_{i} 
=
  \begin{pmatrix}
  0 \\ 
  0 \\
  \vdots \\
  1 \\
  \vdots \\
  0
\end{pmatrix}
\]
Dove l'uno è nella \(i\)-esima riga.
Andiamo a calcolarci la norma 2 di \(A\):
\[\left|\left|A \right|\right|_{2} = \sqrt{\lambda_{MAX}
\left(A^{T} A\right) } \]
Notiamo che per le proprietà della trasposta \(A^{T}= 
\left( U \Sigma V\right)^{T}  =
 V \Sigma^{T} U^{T} \)
\[ \left|\left|A \right|\right|_{2} =
\sqrt{\lambda_{MAX} \left(V 
\Sigma^{T} U^{T} U \Sigma V^{T}\right)} \]
Ora ricordando che \(U\) è ortonormale si ha \(U^{T} U = I\) 
(\(U^{T} = U^{-1}\)):
\[\left|\left|A \right|\right|_{2} 
=
\sqrt{\lambda_{MAX} \left(V \Sigma^{T} \Sigma  V^{T}\right)}\]
Ora ricordando che \(\Sigma\) 
e \(\Sigma^{T}\) sono diagonali si ha 
che vale la proprietà commutativa tra le matrici, dunque possiamo scrivere:
\[\left|\left|A \right|\right|_{2} 
=
\sqrt{\lambda_{MAX} \left(\Sigma^{T} \Sigma V V^{T}\right)}\]
Ora ricordando che \(\Sigma\) è diagonale vale
\(\Sigma^{T} = \Sigma\), dunque si ha:
\[\left|\left|A \right|\right|_{2} = 
\sqrt{\lambda_{MAX} \Sigma^{T} \Sigma }
= \sqrt{\lambda_{MAX} \Sigma \Sigma }
=  \sqrt{\lambda_{MAX} \Sigma^{2}}
\]
Come si vede la norma 2 di \(A\)
è legata al valore singolare maggiore di \(A\).

\begin{figure}[htbp] % [h] suggerisce a LaTeX di mettere la figura "qui" (here)
    \centering
\begin{tikzpicture}[x=2cm, y=2cm]
        
        % --- ASSI (Mantengo le frecce standard qui) ---
        \draw[thick, ->] (-1.8,0) -- (1.8,0) node[right] {$x_1$};
        \draw[thick, ->] (0,-1.8) -- (0,1.8) node[above] {$x_2$};

        % Cerchio unitario (raggio matematico = 1)
        \draw[thick, dashed, black] (0,0) circle (1);
        \node[black, below right] at (-0.7, -0.7) {$r=1$};

        % --- VETTORI CON NUOVE PUNTE ---
        % Nota: Ho cambiato '->' in '-stealth' per le punte dei vettori

        % Vettore x blu, NORMA MATEMATICA 1, Primo quadrante (60 gradi)
        \draw[-stealth, very thick, blue] (0,0) -- (60:1) node[above right] {$\mathbf{x}$};

        % --- MODIFICA QUI ---
        % Vettore y rosso, NORMA MATEMATICA 1.25 (>1)
        % Spostato nel SECONDO QUADRANTE (es. 135 gradi)
        % x1 negativo, x2 positivo
        \draw[-stealth, very thick, red] (0,0) -- (135:2) node[above left] {$\mathbf{y}$};

    \end{tikzpicture}
    \caption{Esempio di rotazione 
    del vettore di ingresso \(x = v_{1}\)
     nel vettore di uscita 
    amplificato \(y = \sigma_{1} u_{1}\). In questo caso abbiamo preso \(m=n=2\),
    dunque la matrice \(A\) è quadrata, mentre \(\Sigma\) è diagonale. Notiamo
    che abbiamo potuto rappresentare tutto sullo stesso grafico solo 
    perché \(m=n\).}
    \label{fig:cerchio_assi}
\end{figure}


\section{Guadagno}
Definiamo gli spazi \(L^{p}\)
come gli spazi delle funzioni \(p\)-sommabili,
cioè in cui esiste la norma \(p\)-esima finita:
\[L^{p}(\mathbb{R^{n}}) = 
\left\{f : \mathbb{R} \to \mathbb{R}^{n} : 
\int_{-\infty}^{+\infty} \left|\left|f(\tau)
\right|\right|^{p}_{p} \, d\tau < \infty \right\}\]
Dove essendo \(f\) una funzione vettoriale a variabile reale
la norma \(p\)-esima è definita come:
\[\left|\left|f(\tau)\right|\right|_{p} =
\left( \sum_{i=1}^{n} |f_{i}(\tau)|^{p} \right)^{\frac{1}{p}}\]
Dove con \(f_{i}(\tau)\) si intende la \(i\)-esima componente del vettore
\(f(\tau)\) (che è un vettore di \(n\) componenti in cui ognuna è una funzione 
di \(\tau\))
\section{Poli e zeri per sistemi MIMO }
% \input{Capitolo2.tex}
% \chapter{Sistemi MIMO nel dominio di Laplace}

Consideriamo il seguente sistema LTI 
a più ingressi e più uscite (MIMO) descritto dalle seguenti equazioni:
\begin{equation}
    \begin{cases}
        \dot{x}(t) = Ax(t) + Bu(t) \\
        y(t) = Cx(t) + Du(t)
    \end{cases}
\end{equation}
Il sistema è MIMO dunque \( u \in \mathbb{R}^{m \times 1}\),
\(y \in \mathbb{R}^{p \times 1}\),
\(B \in \mathbb{R}^{n \times m}\),
\(C \in \mathbb{R}^{p \times n}\)
e \(D \in \mathbb{R}^{p \times m}\).
Il sistema può essere rappresentato nel dominio di Laplace come:
\begin{equation}
    Y(s) = G(s)U(s) 
\end{equation}
Con \( G(s) \in \mathbb{R}^{p \times m} \) matrice di trasferimento del sistema
definita come:
\begin{equation}
    G(s) = C(sI - A)^{-1}B + D
\end{equation}
La matrice \(G(s)\) è composta da \(p \times m\) funzioni di trasferimento
singole \(G_{ij}(s)\):
\[G(s) = \begin{pmatrix}
    G_{11}(s) & \ldots & G_{1m}(s) \\
    \vdots & \vdots & \vdots \\
    G_{p1}(s) & \ldots & G_{pm}(s)
\end{pmatrix}\]
dove ogni elemento \(G_{ij}(s)\) è una funzione polinomiale 
fratta che rappresenta la funzione di trasferimento
tra l'ingresso \(u_j\) e l'uscita \(y_i\).


\section{SVD (Singular Value Decomposition)}
Consideriamo una matrice \(A \in \mathbb{R}^{m \times n}\).
Inoltre dati due vettori \(y \in \mathbb{R}^{m}\) 
e \(x \in \mathbb{R}^{n}\) tali che vale la relazione:
\[y = A x\]
\[A = U \Sigma V^{T}\]
\begin{itemize}
  \item La matrice \(U \in \mathbb{R}^{m \times m}\) (matrice di rotazione delle uscite) 
  è una matrice ortonormale
  composta dai vettori colonna \(u_{1}, u_{2}, \ldots, u_{m}\),
  si ha:
  \[U  =\begin{pmatrix}
    u_{1} & u_{2} & \ldots & u_{m}
  \end{pmatrix}\]
  \item La matrice \(V \in \mathbb{R}^{n \times n}\) 
  (matrice di rotazione degli ingressi) 
  è una matrice ortonormale
  composta dai vettori colonna \(v_{1}, v_{2}, \ldots, v_{n}\),
  si ha:
  \[V  =\begin{pmatrix}
    v_{1} & v_{2} & \ldots & v_{n}
  \end{pmatrix}\]
  \item La matrice \(\Sigma 
  \in \mathbb{R}^{m \times n}\) (matrice dei valori singolari) 

  \[p= min\left\{m,n\right\}\]
   Dove si ha che:
  \[\Sigma =  \begin{pmatrix}
    \Sigma_{1} & 0
  \end{pmatrix} , \hspace{10pt}
    \textnormal{ se } p=m
  \]
  \[\Sigma = \begin{pmatrix}
    \Sigma_{1} \\ 0
  \end{pmatrix} , \hspace{10pt}
    \textnormal{ se } p=n
  \]
\end{itemize}


\[y = \begin{pmatrix}
  u_{1} & u_{2} & \ldots & u_{m} 
\end{pmatrix}
  \begin{pmatrix}
    \Sigma_{1} & 0 & \ldots & 0 \\
    0 & \Sigma_{2} & \ldots & 0 \\
    \vdots & \vdots & \ddots & \vdots \\
    0 & 0 & \ldots & \Sigma_{m} \\
  \end{pmatrix}
\begin{pmatrix}
  v_{1}^{T} \\ 
  v_{2}^{T} \\
  \vdots \\
  v_{n}^{T}
\end{pmatrix}
 \cdot x
\]

Se io prendo \(x=v_{i}\) si ha:
\[
  \begin{pmatrix}
  v_{1}^{T} \\ 
  v_{2}^{T} \\
  \vdots \\
  v_{n}^{T}
\end{pmatrix}
\cdot 
x =
  \begin{pmatrix}
  v_{1}^{T} \\ 
  v_{2}^{T} \\
  \vdots \\
  v_{n}^{T}
\end{pmatrix}
\cdot 
v_{i} 
\]
Ricordando che \(V\) è ortonormale, si che le sue colonne 
sono tra loro ortogonali e di norma 1, dunque si ha:
\[ v^{T}_{j} \cdot v_{i} = 0 , \hspace{10pt} \textnormal{se } i \neq j \]
\[ v^{T}_{j} \cdot v_{i} = 1 , \hspace{10pt} \textnormal{se } i = j \]
Dunque si ha:
\[
  \begin{pmatrix}
  v_{1}^{T} \\ 
  v_{2}^{T} \\
  \vdots \\
  v_{i}^{T} \\
  \vdots \\
  v_{n}^{T}
\end{pmatrix}
\cdot 
v_{i} 
=
  \begin{pmatrix}
  0 \\ 
  0 \\
  \vdots \\
  1 \\
  \vdots \\
  0
\end{pmatrix}
\]
Dove l'uno è nella \(i\)-esima riga.
Andiamo a calcolarci la norma 2 di \(A\):
\[\left|\left|A \right|\right|_{2} = \sqrt{\lambda_{MAX}
\left(A^{T} A\right) } \]
Notiamo che per le proprietà della trasposta \(A^{T}= 
\left( U \Sigma V\right)^{T}  =
 V \Sigma^{T} U^{T} \)
\[ \left|\left|A \right|\right|_{2} =
\sqrt{\lambda_{MAX} \left(V 
\Sigma^{T} U^{T} U \Sigma V^{T}\right)} \]
Ora ricordando che \(U\) è ortonormale si ha \(U^{T} U = I\) 
(\(U^{T} = U^{-1}\)):
\[\left|\left|A \right|\right|_{2} 
=
\sqrt{\lambda_{MAX} \left(V \Sigma^{T} \Sigma  V^{T}\right)}\]
Ora ricordando che \(\Sigma\) 
e \(\Sigma^{T}\) sono diagonali si ha 
che vale la proprietà commutativa tra le matrici, dunque possiamo scrivere:
\[\left|\left|A \right|\right|_{2} 
=
\sqrt{\lambda_{MAX} \left(\Sigma^{T} \Sigma V V^{T}\right)}\]
Ora ricordando che \(\Sigma\) è diagonale vale
\(\Sigma^{T} = \Sigma\), dunque si ha:
\[\left|\left|A \right|\right|_{2} = 
\sqrt{\lambda_{MAX} \Sigma^{T} \Sigma }
= \sqrt{\lambda_{MAX} \Sigma \Sigma }
=  \sqrt{\lambda_{MAX} \Sigma^{2}}
\]
Come si vede la norma 2 di \(A\)
è legata al valore singolare maggiore di \(A\).

\begin{figure}[htbp] % [h] suggerisce a LaTeX di mettere la figura "qui" (here)
    \centering
\begin{tikzpicture}[x=2cm, y=2cm]
        
        % --- ASSI (Mantengo le frecce standard qui) ---
        \draw[thick, ->] (-1.8,0) -- (1.8,0) node[right] {$x_1$};
        \draw[thick, ->] (0,-1.8) -- (0,1.8) node[above] {$x_2$};

        % Cerchio unitario (raggio matematico = 1)
        \draw[thick, dashed, black] (0,0) circle (1);
        \node[black, below right] at (-0.7, -0.7) {$r=1$};

        % --- VETTORI CON NUOVE PUNTE ---
        % Nota: Ho cambiato '->' in '-stealth' per le punte dei vettori

        % Vettore x blu, NORMA MATEMATICA 1, Primo quadrante (60 gradi)
        \draw[-stealth, very thick, blue] (0,0) -- (60:1) node[above right] {$\mathbf{x}$};

        % --- MODIFICA QUI ---
        % Vettore y rosso, NORMA MATEMATICA 1.25 (>1)
        % Spostato nel SECONDO QUADRANTE (es. 135 gradi)
        % x1 negativo, x2 positivo
        \draw[-stealth, very thick, red] (0,0) -- (135:2) node[above left] {$\mathbf{y}$};

    \end{tikzpicture}
    \caption{Esempio di rotazione 
    del vettore di ingresso \(x = v_{1}\)
     nel vettore di uscita 
    amplificato \(y = \sigma_{1} u_{1}\). In questo caso abbiamo preso \(m=n=2\),
    dunque la matrice \(A\) è quadrata, mentre \(\Sigma\) è diagonale. Notiamo
    che abbiamo potuto rappresentare tutto sullo stesso grafico solo 
    perché \(m=n\).}
    \label{fig:cerchio_assi}
\end{figure}


\section{Guadagno}
Definiamo gli spazi \(L^{p}\)
come gli spazi delle funzioni \(p\)-sommabili,
cioè in cui esiste la norma \(p\)-esima finita:
\[L^{p}(\mathbb{R^{n}}) = 
\left\{f : \mathbb{R} \to \mathbb{R}^{n} : 
\int_{-\infty}^{+\infty} \left|\left|f(\tau)
\right|\right|^{p}_{p} \, d\tau < \infty \right\}\]
Dove essendo \(f\) una funzione vettoriale a variabile reale
la norma \(p\)-esima è definita come:
\[\left|\left|f(\tau)\right|\right|_{p} =
\left( \sum_{i=1}^{n} |f_{i}(\tau)|^{p} \right)^{\frac{1}{p}}\]
Dove con \(f_{i}(\tau)\) si intende la \(i\)-esima componente del vettore
\(f(\tau)\) (che è un vettore di \(n\) componenti in cui ognuna è una funzione 
di \(\tau\))
\section{Poli e zeri per sistemi MIMO }
Ora ci poniamo il problema di definire poli e zeri per sistemi MIMO.
Per i sistemi SISO i poli sono definiti come gli zeri del denominatore della 
funzione di trasferimento, mentre gli zeri sono definiti come gli zeri del numeratore
della funzione di trasferimento. Il problema è che per i sistemi MIMO
ho una matrice di trasferimento \(G(s)\) composta da \(pm\) funzioni 
di trasferimento singole \(G_{ij}(s)\), quindi bisogna dare una nuova 
definizione di poli e zeri per sistemi MIMO.
Per prima cosa diamo la definizione di polo per sistemi MIMO.

\begin{definizione}
    Un polo di un sistema MIMO è un valore di \(s\) tale che la matrice \(G(s)\)
    diventa singolare, cioè il suo determinante è nullo, dunque per quel valore di \(s\)
    la matrice \(G(s)\) non è invertibile.
\end{definizione}


\begin{definizione}
    Si definisce rango nominale di una matrice di polinomi 
    \(G(s)\), il rango della matrice che si ha per tutti i valori di 
    \(s\) eccetto un numero finito di valori.
\end{definizione}


\begin{definizione}
    Uno zero di un sistema MIMO è un valore di \(s\) tale 
    che la matrice \(G(s)\)
    perde rango, cioè il rango della matrice \(G(s)\) 
    diminuisce rispetto al suo rango nominale (quello che ha per 
    tutti gli altri valori di \(s\)).
\end{definizione}


% \chapter{Assegnamento degli autovalori con feedback di stato}



\section{Caso MIMO}
Consideriamo un sistema LTI descritto dalle equazioni di stato:
\[
\dot{x}(t) = Ax(t) + Bu(t)
\]
con \(A \in \mathbb{R}^{n \times n}\), 
\(B \in \mathbb{R}^{n \times m}\), 
\(x \in \mathbb{R}^n\) e \(u \in \mathbb{R}^m\). 
Il nostro obiettivo  è attraverso un feedback di stato del tipo:
\begin{equation}
    u(t) = Kx(t)
\end{equation}
con \(K \in \mathbb{R}^{m \times n}\)
(dunque \(Kx \in \mathbb{R}^{ m\times 1}\)) riuscire ad 
assegnare gli autovalori del sistema con matrice della dinamica:
\begin{equation}
    A+BK
\end{equation}
indicheremo l'insieme degli autovalori 
di una matrice \(A\) con \(\sigma(A)\),
dunque noi vogliamo scegliere \(K\) in modo tale da 
poter assegnare l'insieme \(\sigma(A+BK)\) a piacere:
\[\sigma(A+BK) = 
\left\{\lambda_1, \lambda_2, \ldots, \lambda_n
\right\}\]
Per i sistemi MIMO esistono infinite soluzioni al problema di assegnamento 
degli autovalori, cioè esistono infinite matrici \(K\)
tali che l'insieme degli autovalori di \(A+BK\) sia uguale
all'insieme desiderato. Noi andremo a vedere delle soluzioni operative che 
ci permettono di trovare alcune di queste infinite matrici \(K\) soluzioni del problema,
nello specifico vedremo i seguenti 3 metodi:
\begin{itemize}
    \item Metodo della forma canonica di controllabilità
    \item Metodo dell'equazione di Sylvester
    \item Metodo di Kautsky-Nichols-Van Dooren
\end{itemize}

\section{Metodo della forma canonica di controllabilità}
Per sistemi MIMO se la coppia \((A,B)\) è 
completamente controllabile allora è possibile trovare 
una matrice di trasformazione \(T\) che porta il sistema
in una forma canonica di controllabilità a blocchi,
detta forma di Brunovsky:
\[
z = T^{-1}x \Longleftrightarrow x = Tz 
\]

Con le matrici del sistema nella nuova base che valgono:
\begin{equation}
    A_{c} = T^{-1}AT  =
    \begin{pmatrix}
        A_{c1} & 0 & \cdots & 0 \\
        0 & A_{c2} & \cdots & 0 \\
        \vdots & \vdots & \ddots & \vdots \\
        0 & 0 & \cdots & A_{cm}
    \end{pmatrix}
    , \quad B_{c} = T^{-1}B
    = \begin{pmatrix}
        B_{c1} & 0 & \cdots & 0 \\
        0 & B_{c2} & \cdots & 0 \\
        \vdots & \vdots & \ddots & \vdots \\
        0 & 0 & \cdots & B_{cm}
\end{pmatrix}
\end{equation}
Con \(A_{ci} \in \mathbb{R}^{n_{i} \times n_{i}}\)
e \(B_{ci} \in \mathbb{R}^{n_{i} \times 1}\) che valgono:
\[
A_{ci} =
\begin{pmatrix}
    0 & 1 & 0 & \cdots & 0 \\
    0 & 0 & 1 & \cdots & 0 \\
    \vdots & \vdots & \vdots & \ddots & \vdots \\
    -a_{i1} & -a_{i2} & -a_{i3} & \cdots & -a_{in_i} \\
\end{pmatrix}
, \quad
B_{ci} =
\begin{pmatrix}
    0 \\
    0 \\
    \vdots \\
    0 \\
    1
\end{pmatrix}
\]
Inoltre si ha che \(n_{1} + n_{2} + \dots + n_{m} = n\). 
Notiamo che ogni coppia \((A_{ci}, B_{ci})\) rappresenta una 
forma canonica di controllabilità rispetto all'ingresso \(u_{i}\).
Notiamo 
che per come è scritta la matrice \(A_{c}\) con una retroazione 
di stato del tipo:
\[u = K_{c}z\]
con \(K_{c} \in \mathbb{R}^{m \times n}\):
\[
K_{c} =
\begin{pmatrix}
    k_{c1} & 0 & \cdots & 0 \\
    0 & k_{c2} & \cdots & 0 \\
    \vdots & \vdots & \ddots & \vdots \\
    0 & 0 & \cdots & k_{cm}
\end{pmatrix}
\]
con \(k_{ci} \in \mathbb{R}^{1 \times n_{i}}\).
Con \(K_{c}\) scritto in questo modo 
si ha che la matrice della dinamica del sistema 
in retroazione di stato è:
\[
A_{c} + B_{c}K_{c} =
\begin{pmatrix}
    A_{c1} & 0 & \cdots & 0 \\
    0 & A_{c2} & \cdots & 0 \\
    \vdots & \vdots & \ddots & \vdots \\
    0 & 0 & \cdots & A_{cm}
\end{pmatrix}
+
 \begin{pmatrix}
        B_{c1} & 0 & \cdots & 0 \\
        0 & B_{c2} & \cdots & 0 \\
        \vdots & \vdots & \ddots & \vdots \\
        0 & 0 & \cdots & B_{cm}
\end{pmatrix}
\begin{pmatrix}
    k_{c1} & 0 & \cdots & 0 \\
    0 & k_{c2} & \cdots & 0 \\
    \vdots & \vdots & \ddots & \vdots \\
    0 & 0 & \cdots & k_{cm}
\end{pmatrix}
\]
Notando che \(B_{ci} k_{ci} \in \mathbb{R}^{n_{i} \times n_{i}}\)
si ha che:
\[
A_{c} + B_{c}K_{c} =
\begin{pmatrix}
    A_{c1} + B_{c1}k_{c1} & 0 & \cdots & 0 \\
    0 & A_{c2} + B_{c2}k_{c2} & \cdots & 0 \\
    \vdots & \vdots & \ddots & \vdots \\
    0 & 0 & \cdots & A_{cm} + B_{cm}k_{cm}
\end{pmatrix}
\]
Ma la matrice \(A_{c} + B_{c} + K_{c}\)
è diagonale a blocchi dunque si ha che
gli autovalori della matrice sono gli autovalori
dei blocchi diagonali:
\[
\sigma(A_{c} + B_{c}K_{c}) =
\bigcup_{i=1}^{m}
\sigma(A_{ci} + B_{ci}k_{ci})
\]
Ma ogni singolo blocco diagonale 
\(A_{ci} + B_{ci}k_{ci}\) rappresenta un sistema SISO
in forma canonica di controllabilità, e la coppia \((A_{ci}, B_{ci})\)
è completamente controllabile per ipotesi (essendo 
controllabile il sistema originale \((A,B)\)),
dunque per il teorema di assegnamento degli autovalori
applicato ai sistemi SISO esiste \(k_{ci}\)
tale che possiamo assegnare gli autovalori
di ogni blocco diagonale \(A_{ci} + B_{ci}k_{ci}\).
Questo \(k_{ci}\) si trova imponendo
che il polinomio caratteristico del blocco
\(A_{ci} + B_{ci}k_{ci}\)
sia uguale al polinomio caratteristico
desiderato:
\[\det(sI - (A_{ci} + B_{ci}k_{ci})) =
    s^{n_{i}} + \alpha_{i1}s^{n_{i}-1} +
     \alpha_{i2}s^{n_{i}-2} + \ldots + \alpha_{in_{i}}
\]
Ovviamente questo va fatto \(\forall i=1, \dots , m\).
Una volta trovati tutti i \(k_{ci}\)
possiamo costruire la matrice \(K_{c}\), 
per riportarla nella base originale del sistema
basta ricordare:
\[u=K_{c} z= K_{c} T^{-1}x \Longrightarrow 
K= K_{c}T^{-1}\]


\section{Metodo dell'equazione di Sylvester}
Un altro metodo per l'assegnamento degli autovalori 
per sistemi MIMO è il metodo dell'equazione di Sylvester.
L'equazione di Sylvester è la seguente equazione matriciale:
\begin{equation}
    AX + XB = C
\end{equation}
in cui \(A \in \mathbb{R}^{n \times n}\),
\(B \in \mathbb{R}^{m \times m}\) e \(C \in \mathbb{R}^{n \times m}\)
sono matrici note, mentre \(X \in \mathbb{R}^{n \times m}\)
è la matrice incognita da trovare. Andiamo ad analizzare 
questa equazione:
\begin{itemize}
    \item \textbf{Unicità della soluzione}: L'equazione di Sylvester ammette
    una soluzione unica se e solo se gli spettri (cioè 
    gli insiemi degli autovalori) delle matrici \(A\) e \(-B\)
    sono disgiunti, cioè hanno intersezione uguale all'insieme vuoto:
    \[\sigma(A) \cap \sigma(-B) = \emptyset \Longleftrightarrow 
    \lambda_{i}(A) + \lambda_{j}(B) \neq 0 \quad \forall i,j\]
    in parole povere le matrici \(A\) e \(B\) non devono avere autovalori
    con segni opposti, cioè \(A\) e \(-B\) non devono avere autovalori in comune.
    Ad esempio se \(A\) ha un autovalore \(\lambda_{A} = 2\) e \(B\) ha un autovalore 
    \(\lambda_{B} = -2\) allora l'equazione di Sylvester non ammette una soluzione unica.
    \item \textbf{Esistenza della soluzione}: L'equazione di Sylvester ammette almeno
    una soluzione se e solo se:
    \[vec(C) \in Im \left(I_{m} \otimes A + B^{T} \otimes I_{n}\right)\]
    cioè se la matrice \(C\) appartiene all'immagine dell'operatore \(L(X)\):
    \[L(X) = AX + XB\]
    cioè esiste una matrice \(X \in \mathbb{R}^{n \times m}\) tale che \(L(X) = C\).
\end{itemize}

Ora torniamo al nostro problema di assegnamento degli autovalori
per sistemi MIMO. Ricordiamo che il nostro obiettivo è trovare 
la matrice \(K\) tale che:
\[\sigma(A + BK) =
\left\{\lambda_{1}, \lambda_{2}, \ldots, \lambda_{n}\right\}\]
Ma se esiste questa \(K\) allora costruita una matrice \(T\)
che ha sulle colonne gli autovettori associati agli autovalori
desiderati si ha che per 
la definizione di autovettore e autovalore (poichè
il prodotto \((A+BK)T\) equivale a moltiplicare la matrice
\(A+BK\) per ogni autovettore sulle colonne di \(T\)):
\[(A + BK)T = T \Lambda \]
con \(\Lambda\) matrice diagonale(o a blocchi) con gli autovalori desiderati sulla diagonale.
Andando a sviluppare i calcoli si hanno
\[AT + BKT = T \Lambda \]
Portando il termine \(T \Lambda\) a sinistra e \(BKT\) a destra si ha:
\[AT - T \Lambda = - BKT \]
In cui ponendo \(F=-KT\) si ha:
\[AT - T \Lambda = B F \]
Questa è un'equazione di Sylvester
(a meno del segno del secondo termine a primo membro) in cui \(X=T\).
Dunque se riusciamo a risolvere l'equazione di Sylvester
troviamo una \(X\) che ci permette di ricavare una matrice \(K\)
tale che la matrice \(A+BK\) abbia gli autovalori desiderati,
la matrice \(K\) si ricava come:
\[F=-KT=-KX \Longrightarrow K=-F X^{-1} = -F T^{-1}\]


Nella pratica per assegnare gli autovlari tramite l'equazione di Sylvester
si seguono i seguenti passi:
\begin{enumerate}
    \item Si scelgono gli autovalori desiderati
    \(\left\{\lambda_{1}, \lambda_{2}, \ldots, \lambda_{n}\right\}\)
    e si costruisce la matrice \(\Lambda\) con questi autovalori
    sulla diagonale (o a blocchi).
    \item Si costruisce una matrice \(F \in \mathbb{R}^{m \times n}\)
    a piacere (ad esempio con elementi casuali).
    \item Si risolve l'equazione di Sylvester:
    \[AX - X \Lambda = B F \]
    la soluzione \(X\) trovata è proprio la matrice \(T\) degli autovettori
    associati agli autovalori desiderati.
    \item Si calcola la matrice \(K\) come:
    \[K = - F X^{-1} = - F T^{-1}\]
\end{enumerate}
Ovviamente al variare della matrice \(F\) si ottengono
differenti matrici \(K\) che risolvono il problema, dunque 
è sulla matrice \(F\) che possiamo agire per ottenere
differenti soluzioni del problema di assegnamento degli autovalori.
Andiamo ora a definire il condizionamento di una matrice.
    A noi interessa anche che la matrice \(K\) non vari troppo al variare
delle matrici \(A\), \(B\) e \(\Lambda\), dunque ci interessa
che trovare un parametro per valutare la sensibilità numerica della matrice \(K\)
al variare delle matrici \(A\), \(B\) e \(\Lambda\). Ovviamente la 
sensibilità numerica di \(K\) dipende dalla sensibilità numerica
della matrice \(X\).

\begin{definizione}
Questo parametro prende il nome di condizionamento di una matrice ed è definito come:
\begin{equation}
    cond(X) = \|X\| \|X^{-1}\|
\end{equation}
Maggiore è il condizionamento di una matrice,
maggiore è la sensibilità numerica della matrice, dunque il nostro 
obiettivo è trovare una matrice \(X\) con condizionamento il più basso possibile.
\end{definizione}


Andiamo ora trasformare la condizione di esistenza della soluzione
dell'equazione di Sylvester in una condizione sulla coppia \((A,B)\).
Nello specifico la soluzione dell'equazione di Sylvester esiste ed è unica se e solo se
la coppia \((A,B)\) è completamente raggiungibile.
Infatti, se la coppia \((A,B)\) non è completamente raggiungibile,
il sistema può essere portato nella forma di Kalman 
di controllabilità, in cui \(\bar{A}\) e 
\(\bar{B}\) valgono:
\[
\bar{A} =
\begin{pmatrix}
    A_{C} & A_{12} \\
    0 & A_{NC}
\end{pmatrix}
, \quad
\bar{B} =
\begin{pmatrix}
    B_{C} \\
    0
\end{pmatrix}
\]
dove:
\begin{itemize}
    \item La coppia \((A_{C}, B_{C})\) è completamente controllabile
    \item \(A_{NC}\) rappresenta la parte non raggiungibile del sistema.
\end{itemize}
Quindi se andiamo a scrivere l'equazione di Sylvester
per \(\bar{A}\) e \(\bar{B}\) si ha:
\[\bar{A}X - X \Lambda = \bar{B} F \]
Dove le equazioni nelle righe di \(\bar{A}\) e \(\bar{B}\)
in cui il sistema non è controllabile si riducono a:
\[A_{NC} X_{NC} - X_{NC} \Lambda = 0 \]
Ma l'ultima equazione è soddisfatta solo se gli autovalori 
di \(A_{NC}\) e \(\Lambda\) sono uguali, cioè 
se non cambiamo gli autovalori della parte non controllabile del sistema.
Facciamo ora una considerazione sulla scelta degli autovalori.
Se siamo obbligati a scegliere autovalori di \(\Lambda\)
uguali ad autovalori di \(A\),per soddisfare delle specifiche di progetto,
 allora l'equazione di Sylvester non ammette più \textbf{soluzione unica},
e i casi possibili sono due:
\begin{itemize}
    \item L'equazione di Sylvester non ammette soluzioni (il sistema non è raggiungibile)
    \item L'equazione di Sylvester ammette infinite soluzioni (il sistema è raggiungibile)
\end{itemize}
Nel primo caso abbiamo che non esiste una \(X\) che risolve l'equazione
di Sylvester, dunque non possiamo seguire questa strada per trovare \(K\).
Nel secondo caso invece esistono infinite matrici \(X\), dunque infinite
matrici \(K\) che risolvono il problema di assegnamento degli autovalori.
Per garantire un condizionamento migliore del problema invece di scegliere 
autovalori di \(\Lambda\) uguali ad autovalori di \(A\) 
è meglio scegliere autovalori di \(\Lambda\) vicini, ma non proprio uguali 
agli autovalori di \(A\).

\section{Metodo di Kautsky-Nichols-Van Dooren}
Il metodo di Kautsky-Nichols-Van Dooren è un metodo iterativo 
per ottenere una matrice \(K\) ,che risolve il problema di assegnamento
degli autovalori per sistemi MIMO, con un buon condizionamento. Ma il condizionamento
di \(K\) ricordiamo dipendere dal condizionamento della matrice \(X\)
(matrice di autovettori)
che risolve l'equazione di Sylvester. 
Il metodo di Kautsky-Nichols-Van Dooren consiste dei seguenti passaggi:
\begin{enumerate}
    \item Si scelgono gli autovalori desiderati
    \(\left\{\lambda_{1}, \lambda_{2}, \ldots, \lambda_{n}\right\}\)
    e si costruisce la matrice \(\Lambda\) con questi autovalori
    sulla diagonale (o a blocchi).
    \item Si scelgono i vettori \(f_{i}\) in modo casuale (uno per ogni autovalore)
    \item Per ogni \(i\) si risolve il sistema di equazioni:
    \[ \left(A - \lambda_{i} I\right)x_{i} = -Bf_{i}\]
    Ottendendo così per ogni \(i\) un vettore \(x_{i}\).
    \item Si costruisce la matrice \(X\) con i vettori \(x_{i}\) come colonne:
    \[ X = \begin{pmatrix}
        x_{1} & x_{2} & \ldots & x_{n}
    \end{pmatrix}\]
    \item Si valuta il condizionamento di \(X\):
    \[ cond(X) = \|X\| \|X^{-1}\|\]
    \item Se il condizionamento di \(X\) non è accettabile si 
    ritorna al passo 2 e si scelgono nuovi vettori \(f_{i}\).
    Mentre se il condizionamento è accettabile si procede con il passo successivo.
    \item Si calcola la matrice \(F\) come:
    \[ F = \begin{pmatrix}
        f_{1} & f_{2} & \ldots & f_{n}
    \end{pmatrix}\]
    \item Si calcola la matrice \(K\) come:
    \[ K = - F X^{-1} \]
\end{enumerate}

\section{Diagonalizzare}

Sia dato un sistema LTI descritto dalle equazioni di stato:
\[
\dot{x}(t) = Ax(t) + Bu(t)
\]
quando noi diamo un sistema in questa forma diamo per assodato che il vettore 
\(x\) sia scritto nella base canonica di \(\mathbb{R}^{n}\).
Se la matrice \(A\) è diagonalizzabile, ci poniamo il problema di portare \(x\)
in un nuovo stato \(\hat{x}\) in cui \(A\) è diagonale. 
Dunque cerchiamo una matrice di trasformazione \(T\) tale che:
\[ x = T \hat{x} \Longrightarrow
\hat{A} = T^{-1} A T \textnormal{ è diagonale}\] 
Ma come si può immediatamente notare la matrice \(T\) è la matrice del cambiamento 
di base dalla base canonica alla base che diagonalizza \(A\).
Quindi a noi basta trovare la matrice che permette di portare il vettore \(\hat{x}\),
scritto nella base che diagonalizza \(A\), 
nel vettore \(x\) scritto nella base canonica.
Ma la base che diagonalizza \(A\) è formata dagli autovettori di \(A\),
dunque scegliamo un insieme di autovettori linearmente indipendenti di \(A\):
\[ \hat{B} = \left\{v_{1}, v_{2}, \ldots, v_{n}\right\} \] 
Per calcolare \(T\) dobbiamo ricordare che la matrice del cambiamento di base
si calcola attraverso l'applicazione identità (che è un applicazione lineare),
dunque applichiamo l'identità agli autovettori scelti:
\[ id(\hat{B}) =\left\{v_{1} , v_{2} , \ldots , v_{n}\right\} \]
Ora scomponiamo i vettori rispetto alla base canonica 
ed incolonniamoli come colonne della matrice \(T\)(
la base canonica è molto comoda perchè ci permette di scrivere i vettori
proprio come sono), dunque la matrice \(T\) vale:
\[ T = \begin{pmatrix}
    v_{1} & v_{2} & \ldots & v_{n}
\end{pmatrix}\]
In cui i vettori \(v_{1}, v_{2}, \ldots , v_{n} \in \mathbb{R}^{n \times 1}\)
sono vettori colonna \(n \times 1\), dunque la matrice \(T \in \mathbb{R}^{n \times n}\).





\bibliography{Bibliografia}

\end{document}